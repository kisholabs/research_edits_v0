\documentclass{article}

% if you need to pass options to natbib, use, e.g.:
%     \PassOptionsToPackage{numbers, compress}{natbib}
% before loading tackling_climate_workshop_style

% ready for submission
% \usepackage{tackling_climate_workshop_style}

% to compile a preprint version, e.g., for submission to arXiv, add add the
% [preprint] option:
%     \usepackage[preprint]{tackling_climate_workshop_style}

% to compile a camera-ready version, add the [final] option, e.g.:
%     \usepackage[final]{tackling_climate_workshop_style}

% to avoid loading the natbib package, add option nonatbib:
     \usepackage[nonatbib]{tackling_climate_workshop_style}

\usepackage[utf8]{inputenc} % allow utf-8 input
\usepackage[T1]{fontenc}    % use 8-bit T1 fonts
\usepackage{hyperref}       % hyperlinks
\usepackage{url}            % simple URL typesetting
\usepackage{booktabs}       % professional-quality tables
\usepackage{amsfonts}       % blackboard math symbols
\usepackage{nicefrac}       % compact symbols for 1/2, etc.
\usepackage{microtype}      % microtypography

\title{An Inversion Algorithm of Ice Thickness and InSAR Data for
the State of Friction at the Base of the Greenland Ice Sheet}

% The \author macro works with any number of authors. There are two commands
% used to separate the names and addresses of multiple authors: \And and \AND.
%
% Using \And between authors leaves it to LaTeX to determine where to break the
% lines. Using \AND forces a line break at that point. So, if LaTeX puts 3 of 4
% authors names on the first line, and the last on the second line, try using
% \AND instead of \And before the third author name.

\author{%
  David S.~Hippocampus\thanks{Use footnote for providing further information
    about author (webpage, alternative address)---\emph{not} for acknowledging
    funding agencies.} \\
  Department of Computer Science\\
  Cranberry-Lemon University\\
  Pittsburgh, PA 15213 \\
  \texttt{hippo@cs.cranberry-lemon.edu} \\
  % examples of more authors
   \And
   Coauthor \\
   Affiliation \\
   Address \\
   \texttt{email} \\
   \AND
   Coauthor \\
   Affiliation \\
   Address \\
   \texttt{email} \\
  % \And
  % Coauthor \\
  % Affiliation \\
  % Address \\
  % \texttt{email} \\
  % \And
  % Coauthor \\
  % Affiliation \\
  % Address \\
  % \texttt{email} \\
}

\begin{document}

\maketitle

\begin{abstract}
With the advent of climate change and global warming, ice sheets worldwide have been melting at an alarming rate. The largest contributor to global ice loss is the Greenland Ice Sheet (GrIS), losing over 215 Gt per yr, and accounting for ~10\% of mean global sea level rise since the 1990s. It is imperative to understand what dynamics are causing this ice loss and influencing ice flow in order to successfully project mass changes of the GrIS and associated sea level rise. Prior research has found that basal tractions, which represent stress on the ice caused by a layer in between the ice and the bedrock, play a fundamental role in controlling ice flow, but there is no way to measure these basal tractions as they are buried deep beneath the surface. Our work applies machine learning, ice thickness data, and satellite radar data to perform a geophysical inversion to quantify the forces and distributions of the basal tractions that are holding the Greenland Ice sheet back from flowing into the ocean. We partitioned Greenland into 1000 grid cells and generated basis functions (body-forces) for each cell (2° x 2°) in order to generate the best linear combination to train our model to predict the velocity of ice flow across GrIS. Applying Least Squares Method (LSM), Ridge and LASSO regularization, and trade-off (L-curve) criterion, we attained a model with an R$^2$ fit of 0.973 with an RMSE of 6.36 and MAE of 4.30. This model found that the coastline basis tractions are responsible for keeping the majority of the ice sheet from flowing into the ocean. Our research produced a novel inversion algorithm to successfully predict the surface horizontal velocity field and associated traction distributions of the basal friction of the GrIS. The results of this work enhance knowledge of ice flow by uncovering relationships between basal traction, velocity fields, and ice flux, providing a new method of associating ice loss with sea level changes.
\end{abstract}

\section{Introduction}

With the advent of climate change and global warming, ice sheets worldwide have been melting at an alarming rate. The rate of ice mass loss has increased sixfold from 81 billion tons in the 1990’s to 475 billion tons in the 2010s (The IMBIE Team, 2020). The largest contributor to global ice loss is the Greenland Ice Sheet (GrIS), losing over 215 Gt per yr, and accounting for ~10\% of mean global sea level rise since the 1990s (Stocker \&\ Qin, 2013). Rising sea levels have a wide array of disastrous impacts, including coastal erosion, storm surges, flooding, disease spread, and habitat loss that will only continue to worsen in a warming climate (Pattyn et al., 2018). It is imperative to understand what dynamics are causing this ice loss and influencing ice flow in order to successfully project mass changes of the GrIS and associated sea level rise.

Recent advances in satellite remote sensing systems have produced high-resolution measurements and models of the GrIS, making them an ideal tool for studying motion across large ice sheets. Synthetic Aperture Radar (SAR) satellites employ backscattered radar waves to produce precise topographical and elevation maps. Two-pass Interferometric Synthetic Aperture Radar (inSAR) satellites use radar observations from multiple trips over an area of interest to determine surface motion. Over the past decade, InSAR technologies have yielded several large-scale ice velocity maps of the GrIS and Antarctic with highly accurate ice dynamics data. 

(needs fixing)

We focus on two datasets in this work: 

\begin{enumerate} 

ETOPO1 Global Relief Model, published by the NOAA, that provides ice and bedrock topographical surface models for the GrIs.

Radar Satellite Imagery We also use a dataset produced from ESA Sentinel 1 inSAR satellite imagery acquired from 2016 to 2017 composed of ~1800 scenes. Through feature tracking techniques, we are able to derive a surface velocity map across the GrIs. 


\end{enumerate}





In this study, we focus on BLANK and BLANK in order to investigate the state of friction at the base of the GrIS. 

By inverting these datasets through a novel algorithm, we quantify forces and distributions of basal tractions, and horizontal strain rates that are holding the GrIs back from flowing into the ocean. Initial results show significant correlations between inSAR derived strain patterns and flow rates around the coastline of the ice sheet. (revisit)

\section{Previous Work}

Glaciologists have traditionally classified ice motion as a viscous flow (Morland Steady Flow citation). This motivated prior researchers at Stony Brook university to use the elevations of the GrIs derived from the topographical data set mentioned above (ETOPO1) to generate a gravitational potential energy (GPE) map across the GrIs. The primary focus of the research was establishing a connection between GPE calculations and the rate of viscous ice flow. As a result, (BLANK-ask-Jey) assumed that the ice was moving along a friction-less base. However, when the GPE velocity calculations were made, they vastly overestimated the speeds of the ice velocities compared to the inSAR derived ground truth velocities. The results of this work reinforce the idea that the basal traction's have a major influence over ice motion and ice velocity rates (INSERT CITATION). 

Our work is motivated by these results, and acts as an extension building on this prior research to establish a relationship between the basal friction/traction and the inSAR dervied horizontal velocity rates of the GrIs. 

In recent years, advances in computing ML and AI based approaches have gained interest and traction among gaciologists for their ability to produce accurate climate models across a vast amount of data.

Our approach uses a hybrid model: our ground truth velocity data is used to train a linear regression model to predict horizontal velocities and the output of this model is fed into a physical model to estimate strain rates that capture relationships between the physical variables and the actual ice motion.

\section{Methods}

\subsubsection{Dataset}
Our dataset is sourced from two sources: ETOPO1 for topographical ice sheet measurements and Sentinel-1 InSAR data for velocity fields. Both global-level datasets were parsed using the geopandas library such that all points were on the Greenland Ice Sheet. The topographical measurements of Greenland’s ice and bedrock elevations are used to calculate the thickness of the ice, which is then used to generate gravitational potential energies (GPE) across the entire ice sheet. This data feeds into our geophysical equation:

(insert equation here d=G M + vGPE equation)

Our Goal to find best linear combination (BLANK) that predicts BLANK After achieving a proper fit, the model coefficients would be able to provide valuable insight to the distributions of basal tractions across the Greenland ice sheet. 

\subsubsection{G matrix}
To create our G matrix, we partitioned Greenland into 1000 grid cells (each with size 2* x 2*) and generated 3 basis functions (Exx - Horizontal East and West effective body forces, Eyy - Horizontal North and South effective body-forces, and Exy - Shear effecitve body-forces) for each cell. Our original experiments used 40 grid cells (each 10 x 10), and we noticed considerable performance improvements increasing the resolution of our G matrix. Depending on computational computing power available, further reducing the grid cell size is an avenue for further optimizing the fit of the BLANK linear combination.   

\subsubsection{Model}

This is a linear inversion task, where the effective body-forces in one grid cell can have effects on its surrounding cells and beyond. Thus, it requires a regression model. We use Least Squares Regression from the sklearn library, which has shown to perform well in inversion tasks (Lines & Trei℡, 1984). To further generalize and optimize the overall fit of our model, we also employed regularization methods Ridge and LASSO. Our Loss Function is defined as [INSERT LOSS HERE], with our alpha value range defined as [alpha]. We used the trade-off (L-curve) criterion and the 10-fold generalized cross validation techniques to determine our optimal smoothing parameter. 

\section{Results}

\subsubsection{Model Fit}

Both the Ridge and LASSO regression models achieved a near identical fit to the velocity field, achieving R$^$2 values of 0.999 and 0.987 respectively. The model predicted plotted against the ground truth velocity field has been shown in Figure BLANK, highlighting the accuracy of the predictions. The slight errors in the predictions seem to be mostly located along the coastline, and we believe that these errors can be corrected for through a smaller grid cell size and access to higher computing power. 


\section{Conclusion and Future Work}

This work has large implications on the ability to individually separate basal traction from GPE. It also serves as a step towards modeling ice flux in relation to friction, normal forces, and other nonlinear forces. For example, to quantify the changes in ice flux, we will run multiple inversions and define the velocity field near the edge of Greenland, and we purposely decrease the velocity magnitude by 50\% and that reduced velocity becomes the new data uploaded into the inversion model to get the new basal tractions. We would do a grid search to find a proper velocity field percentage that results in the basal traction reduction.

We are currently looking for more data spanning more time or also for the Antarctic Ice Sheet.


\section*{References}

References follow the acknowledgments. Use unnumbered first-level heading for
the references. Any choice of citation style is acceptable as long as you are
consistent. It is permissible to reduce the font size to \verb+small+ (9 point)
when listing the references.
{\bf Note that the Reference section does not count towards the pages of content that are allowed; 4 pages for Papers track and 3 pages for Proposals track.}
\medskip

\small

[1] Alexander, J.A.\ \& Mozer, M.C.\ (1995) Template-based algorithms for
connectionist rule extraction. In G.\ Tesauro, D.S.\ Touretzky and T.K.\ Leen
(eds.), {\it Advances in Neural Information Processing Systems 7},
pp.\ 609--616. Cambridge, MA: MIT Press.

[2] Bower, J.M.\ \& Beeman, D.\ (1995) {\it The Book of GENESIS: Exploring
  Realistic Neural Models with the GEneral NEural SImulation System.}  New York:
TELOS/Springer--Verlag.

[3] Hasselmo, M.E., Schnell, E.\ \& Barkai, E.\ (1995) Dynamics of learning and
recall at excitatory recurrent synapses and cholinergic modulation in rat
hippocampal region CA3. {\it Journal of Neuroscience} {\bf 15}(7):5249-5262.

\end{document}