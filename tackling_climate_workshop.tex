\documentclass{article}

% if you need to pass options to natbib, use, e.g.:
%     \PassOptionsToPackage{numbers, compress}{natbib}
% before loading tackling_climate_workshop_style

% ready for submission
% \usepackage{tackling_climate_workshop_style}

% to compile a preprint version, e.g., for submission to arXiv, add add the
% [preprint] option:
%     \usepackage[preprint]{tackling_climate_workshop_style}

% to compile a camera-ready version, add the [final] option, e.g.:
%     \usepackage[final]{tackling_climate_workshop_style}

% to avoid loading the natbib package, add option nonatbib:
     \usepackage[nonatbib]{tackling_climate_workshop_style}

\usepackage[utf8]{inputenc} % allow utf-8 input
\usepackage[T1]{fontenc}    % use 8-bit T1 fonts
\usepackage{hyperref}       % hyperlinks
\usepackage{url}            % simple URL typesetting
\usepackage{booktabs}       % professional-quality tables
\usepackage{amsfonts}       % blackboard math symbols
\usepackage{nicefrac}       % compact symbols for 1/2, etc.
\usepackage{microtype}      % microtypography

\title{\textit{Formatting instructions for} Tackling Climate Change with Machine Learning: workshop at NeurIPS 2022}

% The \author macro works with any number of authors. There are two commands
% used to separate the names and addresses of multiple authors: \And and \AND.
%
% Using \And between authors leaves it to LaTeX to determine where to break the
% lines. Using \AND forces a line break at that point. So, if LaTeX puts 3 of 4
% authors names on the first line, and the last on the second line, try using
% \AND instead of \And before the third author name.

\author{%
  David S.~Hippocampus\thanks{Use footnote for providing further information
    about author (webpage, alternative address)---\emph{not} for acknowledging
    funding agencies.} \\
  Department of Computer Science\\
  Cranberry-Lemon University\\
  Pittsburgh, PA 15213 \\
  \texttt{hippo@cs.cranberry-lemon.edu} \\
  % examples of more authors
  % \And
  % Coauthor \\
  % Affiliation \\
  % Address \\
  % \texttt{email} \\
  % \AND
  % Coauthor \\
  % Affiliation \\
  % Address \\
  % \texttt{email} \\
  % \And
  % Coauthor \\
  % Affiliation \\
  % Address \\
  % \texttt{email} \\
  % \And
  % Coauthor \\
  % Affiliation \\
  % Address \\
  % \texttt{email} \\
}

\begin{document}

\maketitle

\begin{abstract}
With the advent of climate change and global warming, ice sheets worldwide have been melting at an alarming rate. The largest contributor to global ice loss is the Greenland Ice Sheet (GrIS), losing over 215 Gt per yr, and accounting for ~10\% of mean global sea level rise since the 1990s. It is imperative to understand what dynamics are causing this ice loss and influencing ice flow in order to successfully project mass changes of the GrIS and associated sea level rise. Prior research has found that basal tractions, which represent stress on the ice caused by a layer in between the ice and the bedrock, play a fundamental role in controlling ice flow, but there is no way to measure these basal tractions as they are buried deep beneath the surface. Our work applies machine learning, ice thickness data, and satellite radar data to perform a geophysical inversion to quantify the forces and distributions of the basal tractions that are holding the Greenland Ice sheet back from flowing into the ocean. We partitioned Greenland into 1000 grid cells and generated basis functions (body-forces) for each cell (2° x 2°) in order to generate the best linear combination to train our model to predict the velocity of ice flow across GrIS. Applying Least Squares Method (LSM), Ridge and LASSO regularization, and trade-off (L-curve) criterion, we attained a model with an R$^2$ fit of 0.973 with an RMSE of 6.36 and MAE of 4.30. This model found that the coastline basis tractions are responsible for keeping the majority of the ice sheet from flowing into the ocean. Our research produced a novel inversion algorithm to successfully predict the surface horizontal velocity field and associated traction distributions of the basal friction of the GrIS. The results of this work enhance knowledge of ice flow by uncovering relationships between basal traction, velocity fields, and ice flux, providing a new method of associating ice loss with sea level changes.
\end{abstract}

\section{Introduction}

With the advent of climate change and global warming, ice sheets worldwide have been melting at an alarming rate. The rate of ice mass loss has increased sixfold from 81 billion tons in the 1990’s to 475 billion tons in the 2010s (The IMBIE Team, 2020). The largest contributor to global ice loss is the Greenland Ice Sheet (GrIS), losing over 215 Gt per yr, and accounting for ~10\% of mean global sea level rise since the 1990s (Stocker & Qin, 2013). Rising sea levels have a wide array of disastrous impacts, including coastal erosion, storm surges, flooding, disease spread, and habitat loss that will only continue to worsen in a warming climate (Pattyn et al., 2018). It is imperative to understand what dynamics are causing this ice loss and influencing ice flow in order to successfully project mass changes of the GrIS and associated sea level rise.

Recent advances in satellite remote sensing systems have produced high-resolution measurements and models of the GrIS, making them an ideal tool for studying motion across large ice sheets. Synthetic Aperture Radar (SAR) satellites employ backscattered radar waves to produce precise topographical and elevation maps. Two-pass Interferometric Synthetic Aperture Radar (inSAR) satellites use radar observations from multiple trips over an area of interest to determine surface motion. Over the past decade, InSAR technologies has yielded several large-scale ice velocity maps of the GrIS and Antarctic with highly accurate ice dynamics data.


\section{Previous Work}

Electronic submissions are required, via this submission website:
\begin{center}
  \url{https://cmt3.research.microsoft.com/CCAINeurIPS2022/}
\end{center}

Please read the instructions below carefully and follow them faithfully.

\section{Methods}

There are three tracks for submissions: \textbf{Papers, Proposals, and Tutorial}s, each described in detail below.  Submissions are limited to \textbf{4 pages for the Papers track}, and \textbf{3 pages for the Proposals track}. Tutorials do not have a page limit. References do not count towards this total. Supplementary appendices are allowed but will be read at the discretion of the reviewers. All submissions must explain why the proposed work has (or could have) positive impacts regarding climate change.

\subsubsection{Dataset}

\subsubsection{Model}

\section{Results}

\section{Conclusion}

\subsubsection{Future Works}



\textit{Work that is in progress, published, and/or deployed.}

Submissions for the Papers track should describe projects relevant to climate change that involve machine learning. These may include (but are not limited to) academic research; deployed results from startups, industry, public institutions, etc.; and climate-relevant datasets.

Submissions should provide experimental or theoretical validation of the method presented, as well as specifying what gap the method fills. Authors should clearly illustrate a pathway to climate impact, i.e., identify the way in which this work fits into broader efforts to address climate change. Algorithms need not be novel from a machine learning perspective if they are applied in a novel setting. Details of methodology need not be revealed if they are proprietary, though transparency is highly encouraged.

Submissions creating novel datasets are welcomed. Datasets should be designed to permit machine learning research (e.g. formatted with clear benchmarks for evaluation). In this case, baseline experimental results on the dataset are preferred, but not required.

Submissions are limited to 4 pages. References do not count toward this total. Submissions are due Sept. 18, 2022.

\subsubsection{Proposals}

\textit{Early-stage work and detailed descriptions of ideas for future work.}

Submissions for the Proposals track should describe detailed ideas for how machine learning can be used to solve climate-relevant problems. While less constrained than the Papers track, Proposals will be subject to a very high standard of review. Ideas should be justified as extensively as possible, including motivation for why the problem being solved is important in tackling climate change, discussion of why current methods are inadequate, explanation of the proposed method, and discussion of the pathway to climate impact. Preliminary results are optional.

Submissions are limited to 3 pages. References do not count toward this total. Submissions are due Sept. 18, 2022.

\subsubsection{Tutorials}
\textit{Interactive notebooks for insightful step-by-step walkthroughs.}

Submissions for the Tutorials track should introduce or demonstrate the use of ML methods and tools such as libraries, packages, services, datasets, or frameworks to address a problem related to climate change. Tutorial proposals (due Aug 18) should take the form of an abstract and should include a clear and concise description of users' expected learning outcomes from the tutorial. Accepted submissions (to be notified by Aug 25) will be given about 3 weeks for the initial tutorial development (midterm deadline on Sep 18), after which tutorial creators will collaborate with the Tutorials Team, who will review the tutorials periodically and provide iterative feedback, while the creators continue to develop and improve their work over the course of another 8 weeks. Midterm tutorial submissions (due Sep 18) and Final tutorial submissions (due Nov 3) should be in the form of executable notebooks (e.g. Jupyter, Colab). Submissions will be reviewed based on their potential impact and overall usability by the climate and AI research community.


\subsection{Style}

Papers must be prepared according to the instructions presented here.   Submissions are limited to \textbf{4 pages for the Papers track}, and \textbf{3 pages for the Proposals track}. Tutorials do not have a page limit. Papers that exceed these page limitswill not be reviewed, or in any other way considered for presentation at the workshop.


Authors are required to use the workshop style files (modified from the NeurIPS style files), obtainable on the website as indicated below. Please make sure you use the current files and not previous versions. Tweaking the style files may be grounds for
rejection.

\subsection{Retrieval of style files}

The style files for this workshop are available on
the World Wide Web at
\begin{center}
  \url{http://climatechange.ai/files/TCCML_NeurIPS_2022_Style_File.zip} %TODO UPDATE THIS
\end{center}
The file \verb+tackling_climate_workshop.pdf+ contains these instructions and illustrates the
various formatting requirements your paper must satisfy.

The file \verb+tackling_climate_workshop.tex+ may be used as a “shell” for writing your paper. Alternatively, the file \verb+tackling_climate_workshop.docx+ can be used as well. Replace the author, title, abstract, and text of the paper with your own. Please remember that at submission time your document should be anonymized and the only accepted format is PDF.

The only supported style file for \LaTeX{} is \verb+tackling_climate_workshop_style.sty+,
rewritten for \LaTeXe{}.  \textbf{Previous style files for \LaTeX{} 2.09, or NeurIPS conference style file, are not accepted.}

The \LaTeX{} style file contains three optional arguments: \verb+final+, which
creates a camera-ready copy, \verb+preprint+, which creates a preprint for
submission to, e.g., arXiv, and \verb+nonatbib+, which will not load the
\verb+natbib+ package for you in case of package clash.

\paragraph{Preprint option}
If you wish to post a preprint of your work online, e.g., on arXiv, using the workshop style, please use the \verb+preprint+ option. This will create a
nonanonymized version of your work with the text ``Preprint. Work in progress.''
in the footer. This version may be distributed as you see fit. Please \textbf{do
  not} use the \verb+final+ option, which should \textbf{only} be used for
papers accepted to the workshop. Note that all workshops are non-archival; submission does not preclude future publication.

At submission time, please omit the \verb+final+ and \verb+preprint+
options. This will anonymize your submission and add line numbers to aid
review. Please do \emph{not} refer to these line numbers in your paper as they
will be removed during generation of camera-ready copies.

The file \verb+tackling_climate_workshop.tex+ may be used as a ``shell'' for writing your
paper. Replace the author, title, abstract, and text of
the paper with your own.

The formatting instructions contained in these style files are summarized in
Sections \ref{gen_inst}, \ref{headings}, and \ref{others} below.

\section{General formatting instructions}
\label{gen_inst}

The text must be confined within a rectangle 5.5~inches (33~picas) wide and
9~inches (54~picas) long. The left margin is 1.5~inch (9~picas).  Use 10~point
type with a vertical spacing (leading) of 11~points.  Times New Roman is the
preferred typeface throughout, and will be selected for you by default.
Paragraphs are separated by \nicefrac{1}{2}~line space (5.5 points), with no
indentation.

The paper title should be 17~point, initial caps/lower case, bold, centered
between two horizontal rules. The top rule should be 4~points thick and the
bottom rule should be 1~point thick. Allow \nicefrac{1}{4}~inch space above and
below the title to rules. All pages should start at 1~inch (6~picas) from the
top of the page.

The version of the paper submitted for review should have "Anonymous Author(s)" as the author of the paper. 
For the final version, authors' names are set in boldface, and each name is
centered above the corresponding address. The lead author's name is to be listed
first (left-most), and the co-authors' names (if different address) are set to
follow. If there is only one co-author, list both author and co-author side by
side.

Please pay special attention to the instructions in Section \ref{others}
regarding figures, tables, acknowledgments, and references.

\section{Headings: first level}
\label{headings}

All headings should be lower case (except for first word and proper nouns),
flush left, and bold.

First level headings are in point size 12. One line space before the first level heading and \nicefrac{1}{2} line space after the first level heading.

\subsection{Headings: second level}

Second level headings are in point size 10. One line space before the second level heading and \nicefrac{1}{2} line space after the second level heading.

\subsubsection{Headings: third level}

Third level headings are in point size 10. One line space before the third level heading and \nicefrac{1}{2} line space after the third level heading. 

\paragraph{Paragraphs}

In \LaTeX{} there is also a \verb+\paragraph+ command available, which sets the heading in bold, flush left, and inline with the text, with the heading followed by 1\,em of space. If using this style option in a \verb+docx+ file, please follow these instructions accordingly.

\section{Citations, figures, tables, references}
\label{others}

These instructions apply to everyone, regardless of the formatter being used.

\subsection{Citations within the text}

Citations within the text should be numbered consecutively.  The corresponding number is to appear enclosed in square brackets, such as [1] or [2]-[5].  The corresponding references are to be listed in the same order at the end of the paper, in the \textbf{References} section. (Note: the standard
\textsc{Bib\TeX} style \texttt{unsrt} produces this.) As to the format of the references themselves, any standard reference style is acceptable, as long as it is used consistently.


As submission is double blind, refer to your own published work in the third
person. That is, use ``In the previous work of Jones et al.\ [4],'' not ``In our
previous work [4].'' If you cite your other papers that are not widely available
(e.g., a journal paper under review), use anonymous author names in the
citation, e.g., an author of the form ``A.\ Anonymous.''

When using the \LaTeX{} template, the \verb+natbib+ package will be loaded for you by default.  Citations may be author/year or numeric, as long as you maintain internal consistency.

For \LaTeX{} use, note that the documentation for \verb+natbib+ may be found at
\begin{center}
  \url{http://mirrors.ctan.org/macros/latex/contrib/natbib/natnotes.pdf}
\end{center}
Of note is the command \verb+\citet+, which produces citations appropriate for
use in inline text.  For example,
\begin{verbatim}
   \citet{hasselmo} investigated\dots
\end{verbatim}
produces
\begin{quote}
  Hasselmo, et al.\ (1995) investigated\dots
\end{quote}

If you wish to load the \verb+natbib+ package with options, you may add the
following before loading the \verb+neurips_2020+ package:
\begin{verbatim}
   \PassOptionsToPackage{options}{natbib}
\end{verbatim}

If \verb+natbib+ clashes with another package you load, you can add the optional
argument \verb+nonatbib+ when loading the style file:
\begin{verbatim}
   \usepackage[nonatbib]{tackling_climate_workshop_style}
\end{verbatim}

\subsection{Footnotes}

Footnotes should be used sparingly.  If you do require a footnote, indicate
footnotes with a number\footnote{Sample of the first footnote.} in the
text. Place the footnotes at the bottom of the page on which they appear.
Precede the footnote with a horizontal rule of 2~inches (12~picas).

Note that footnotes are properly typeset \emph{after} punctuation
marks.\footnote{As in this example.}

\subsection{Figures}

\begin{figure}
  \centering
  \fbox{\rule[-.5cm]{0cm}{4cm} \rule[-.5cm]{4cm}{0cm}}
  \caption{Sample figure caption.}
\end{figure}

All artwork must be neat, clean, and legible. Lines should be dark enough for
purposes of reproduction. The figure number and caption always appear after the
figure. Place one line space before the figure caption and one line space after
the figure. The figure caption should be lower case (except for first word and
proper nouns); figures are numbered consecutively.

Make sure the figure caption does not get separated from the figure. Leave sufficient space to avoid splitting the figure and figure caption.

You may use color figures.  However, it is best for the figure captions and the
paper body to be legible if the paper is printed in either black/white or in
color, and that colormaps consider accessibility to the visually impaired (e.g. red/green colorblindness).

\subsection{Tables}

All tables must be centered, neat, clean and legible.  The table number and
title always appear before the table.  See Table~\ref{sample-table}.

Place one line space before the table title, one line space after the
table title, and one line space after the table. The table title must
be lower case (except for first word and proper nouns); tables are
numbered consecutively.

Note that publication-quality tables \emph{do not contain vertical rules.} We
strongly suggest the use of the \verb+booktabs+ package, which allows for
typesetting high-quality, professional tables:
\begin{center}
  \url{https://www.ctan.org/pkg/booktabs}
\end{center}
This package was used to typeset Table~\ref{sample-table}.

\begin{table}
  \caption{Sample table title}
  \label{sample-table}
  \centering
  \begin{tabular}{lll}
    \toprule
    \multicolumn{2}{c}{Part}                   \\
    \cmidrule(r){1-2}
    Name     & Description     & Size ($\mu$m) \\
    \midrule
    Dendrite & Input terminal  & $\sim$100     \\
    Axon     & Output terminal & $\sim$10      \\
    Soma     & Cell body       & up to $10^6$  \\
    \bottomrule
  \end{tabular}
\end{table}

\section{Final instructions}

Do not change any aspects of the formatting parameters in the style files.  In
particular, do not modify the width or length of the rectangle the text should
fit into, and do not change font sizes (except perhaps in the
\textbf{References} section; see below). Please note that pages should be
numbered.

\section{Preparing PDF files}

Please prepare submission files with paper size ``US Letter,'' and not, for
example, ``A4.''

Fonts were the main cause of problems in the past years. Your PDF file must only
contain Type 1 or Embedded TrueType fonts. Here are a few instructions to
achieve this.

\begin{itemize}

\item For MSWord users: from the print menu, click the PDF drop-down box, and select "Save as PDF…".

\item For \LaTeX{} users: you should directly generate PDF files using \verb+pdflatex+.

\item You can check which fonts a PDF files uses.  In Acrobat Reader, select the
  menu Files$>$Document Properties$>$Fonts and select Show All Fonts. You can
  also use the program \verb+pdffonts+ which comes with \verb+xpdf+ and is
  available out-of-the-box on most Linux machines.

\item The IEEE has recommendations for generating PDF files whose fonts are also
  acceptable for NeurIPS. Please see
  \url{http://www.emfield.org/icuwb2010/downloads/IEEE-PDF-SpecV32.pdf}

\item \verb+xfig+ "patterned" shapes are implemented with bitmap fonts.  Use
  "solid" shapes instead.

\item The \verb+\bbold+ package almost always uses bitmap fonts.  You should use
  the equivalent AMS Fonts:
\begin{verbatim}
   \usepackage{amsfonts}
\end{verbatim}
followed by, e.g., \verb+\mathbb{R}+, \verb+\mathbb{N}+, or \verb+\mathbb{C}+
for $\mathbb{R}$, $\mathbb{N}$ or $\mathbb{C}$.  You can also use the following
workaround for reals, natural and complex:
\begin{verbatim}
   \newcommand{\RR}{I\!\!R} %real numbers
   \newcommand{\Nat}{I\!\!N} %natural numbers
   \newcommand{\CC}{I\!\!\!\!C} %complex numbers
\end{verbatim}
Note that \verb+amsfonts+ is automatically loaded by the \verb+amssymb+ package.

\end{itemize}

If your file contains type 3 fonts or non embedded TrueType fonts, we will ask
you to fix it.

\subsection{Margins in \LaTeX{}}

With \LaTeX{} most of the margin problems come from figures positioned by hand using
\verb+\special+ or other commands. We suggest using the command
\verb+\includegraphics+ from the \verb+graphicx+ package. Always specify the
figure width as a multiple of the line width as in the example below:
\begin{verbatim}
   \usepackage[pdftex]{graphicx} ...
   \includegraphics[width=0.8\linewidth]{myfile.pdf}
\end{verbatim}
See Section 4.4 in the graphics bundle documentation
(\url{http://mirrors.ctan.org/macros/latex/required/graphics/grfguide.pdf})

A number of width problems arise when \LaTeX{} cannot properly hyphenate a
line. Please give LaTeX hyphenation hints using the \verb+\-+ command when
necessary.

\begin{ack}
Use unnumbered first-level headings for the acknowledgments. All acknowledgments
go at the end of the paper before the list of references. Moreover, you are required to declare 
funding (financial activities supporting the submitted work) and competing interests (related financial activities outside the submitted work). 
More information about this disclosure can be found at: \url{https://neurips.cc/Conferences/2020/PaperInformation/FundingDisclosure}.


Do {\bf not} include this section in the anonymized submission, only in the final paper. You can use the \texttt{ack} environment provided in the style file to automatically hide this section in the anonymized submission.
\end{ack}

\section*{References}

References follow the acknowledgments. Use unnumbered first-level heading for
the references. Any choice of citation style is acceptable as long as you are
consistent. It is permissible to reduce the font size to \verb+small+ (9 point)
when listing the references.
{\bf Note that the Reference section does not count towards the pages of content that are allowed; 4 pages for Papers track and 3 pages for Proposals track.}
\medskip

\small

[1] Alexander, J.A.\ \& Mozer, M.C.\ (1995) Template-based algorithms for
connectionist rule extraction. In G.\ Tesauro, D.S.\ Touretzky and T.K.\ Leen
(eds.), {\it Advances in Neural Information Processing Systems 7},
pp.\ 609--616. Cambridge, MA: MIT Press.

[2] Bower, J.M.\ \& Beeman, D.\ (1995) {\it The Book of GENESIS: Exploring
  Realistic Neural Models with the GEneral NEural SImulation System.}  New York:
TELOS/Springer--Verlag.

[3] Hasselmo, M.E., Schnell, E.\ \& Barkai, E.\ (1995) Dynamics of learning and
recall at excitatory recurrent synapses and cholinergic modulation in rat
hippocampal region CA3. {\it Journal of Neuroscience} {\bf 15}(7):5249-5262.

\end{document}